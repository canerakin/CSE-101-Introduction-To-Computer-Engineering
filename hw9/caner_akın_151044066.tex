\documentclass[12pt]{article}
\usepackage[turkish]{babel}
\usepackage[utf8]{inputenc}
\usepackage[T1]{fontenc}
\textheight = 609pt
\begin{document}
\title{BIL 101 \\ HW9}
\author{Caner AKIN}
\date{ 1 Aralık 2017}
\maketitle
\newpage
\section{Soru1}
\subsection{ Reinforcement learning (pekiştirmeli öğrenme) nedir?}
Reinforcement Learning, Machine Learning’in bir alanıdır ve davranış psikolojisinden
yola çıkılarak keşfedilen aynı zamanda yapay zekada da kullanılan bir öğrenme yöntemidir.
Asıl amaç agent'in çevreyle etkileşerek çevreden geri bildirim alır ve aldığı bu geri
bildirime maksimuma çıkartıp bulunmasıdır.Reinforcement Learning algoritmalarının klasik 
tekniklerden farkı,ön bilgiye ihtiyaç duymadan ve kesin yöntemlerin verimsiz kaldığı 
büyükalanlarda kullanılmaktadır.Birçok pekiştirmeli öğrenme algoritması dinamik programlama 
tekniklerini kullanır.
\par
\subsection{ Reinforcement learning'in  diğer makine öğrenmesi \-yöntemlerinden farkı nedir?}
Daha çok amaca odaklı bir öğrenme yöntemi olduğu için Machine Learning'in diğer yaklaşımlarından
ayrılıyor.
\newpage
\section{Soru2}
\subsection{ Görüntü işleme, 2 boyutlu 3 boyutlu grafik tekniklerinin birbirinden farkı nedir?}
Genel olarak görüntü işleme 2 boyutlu görüntülerin kullanacağı zaman görüntülerin
analizini ele alır.Basit bir şekilde 2B grafikler 2 boyutlu şekillerin görüntüye 
dönüştürülmesi ile ilgilenir.Yani görüntü işleme 2 boyutlu görüntülerin fotoğrafların sanal
dünyaya yerleştirilmesidir.3B grafikler içinse 3 boyutlu ortamların veya sahnelerin görüntüye
dönüştürülmesini ele alınır.2 boyutlu şekillerin görüntüye dönüştürülmesinin tersine 3 boyutlu
şekiller,görüntüye dönüştürülmesiyle ilgilenir.Bu süreç 3 boyutlu sahnenin dijital olarak
kodlanmış versiyonunun oluşturulması ve bu sahnelerin görüntülerinin oluşturulması için
fotoğrafik işlemlerin simüle edilmesidir.2 boyutla oluşturulan filmlerde sanal dünyalardaki 
görüntü kopyalanırken 3 boyutlu filmlerde bu sanal dünyaların kendisi sunulmaktadır.
\subsection{3 boyutlu grafik işlemenin 3 temel adımını açıklayınız.}
3 boyutlu grafik işlemenin 3 temel adımı;\\	
-Modelleme(ortamın oluşturulması)	\\
-İmge oluşturma(resim oluşturma)\\
-Görüntüleme(resmin görüntülenmesi)\\
\\Modelleme:\par
Modelleme cismin 3 boyutlu şekilde bütün yüzeyleriyle matematiksel ifadesidir.
3D modeller fiziksel bir varlığı uzay geometrisindeki belirli noktalarla tanımlar.
Bu modeller çeşitli geometrik şekillerin birleşimi ile ortaya çıkar.\\
\\İmge oluşturma:\par
Nesnelerin bir yansımasında nasıl görüneceğini belirlemeyi gerçekleştirir.
Bir nesnenin görünümü o nesneden yayılan ışık tarafından belirlendiğinden nesnenin görünümünü
belirlemek nihayetinde ışığın davranışını simüle etme işlemine dönüşür.\\
\\Görüntüleme:\par
Çerçeve arabelleğinde saklanmış olan görüntü yada gösteriler  yada daha sonra gösterilmek
üzere daha kalıcı bir depolama alanına aktarılır.
\end{document}